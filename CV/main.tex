\documentclass[a4paper,10pt]{article}
\usepackage[utf8]{inputenc}
\usepackage[russian]{babel}
\usepackage[left=2cm,right=2cm,top=2cm,bottom=2cm]{geometry}
\usepackage{enumitem}
\usepackage{titlesec}
\usepackage{parskip}

\titleformat{\section}{\bfseries\Large}{}{0em}{}
\renewcommand{\baselinestretch}{1.1}

\begin{document}

\section*{Лилит Сиепанян}
\noindent
\textbf{Email:} lastepanyan@edu.hse.ru \\
\textbf{Дата рождения:} 20 мая 2003 \\
\textbf{Город:} Ереван \\

\section*{Образование}
\textbf{Высшая школа экономики (НИУ ВШЭ)}, Москва \\
Бакалавриат, Факультет компьютерных наук, КНАД \\
Сентябрь 2022 --- настоящее время

\section*{Курсы и учебные проекты}
\begin{itemize}[leftmargin=*]
  \item Летняя практика “Основы Deep Learning” — работа с PyTorch, построение CNN и RNN моделей.
  \item Алгоритмы и структуры данных (C++, Python)
  \item Курс по машинному обучению (базовый и продвинутый) — регрессия, классификация, ансамбли, детекция аномалий.
  \item \textbf{AI-проект:} разработка и внедрение модели для выявления диабетической ретинопатии по снимкам сетчатки глаза с использованием сверточных нейросетей.
  \item \textbf{ML-проект:} автогенерация обложек для видео на основе видеоряда и текстового описания — использование моделей Vision+Text (CLIP/BLIP) для автоматизации дизайна.
\end{itemize}

\section*{Навыки}
\begin{itemize}[leftmargin=*]
  \item \textbf{Языки программирования:} Python, C++
  \item \textbf{ML/AI:} sklearn, numpy, pandas, PyTorch
  \item \textbf{Инструменты:} Git, Docker, Linux, VSCode
\end{itemize}

\section*{Языки}
\begin{itemize}[leftmargin=*]
  \item Русский — свободно на уровне носителя
  \item Английский — B2 (читаю техническую документацию)
  \item Армянский — родной
\end{itemize}

\section*{О себе}
Ответственная, быстро учусь, люблю разбираться в новых технических темах. Сейчас активно изучаю области ML и системного программирования.

\end{document}
\documentclass[a4paper,10pt]{article}
\usepackage[utf8]{inputenc}
\usepackage[russian]{babel}
\usepackage[left=2cm,right=2cm,top=2cm,bottom=2cm]{geometry}
\usepackage{enumitem}
\usepackage{titlesec}
\usepackage{parskip}

\titleformat{\section}{\bfseries\Large}{}{0em}{}
\renewcommand{\baselinestretch}{1.1}

\begin{document}

\section*{Лилит Сиепанян}
\noindent
\textbf{Email:} lastepanyan@edu.hse.ru \\
\textbf{Дата рождения:} 20 мая 2003 \\
\textbf{Город:} Ереван \\

\section*{Образование}
\textbf{Высшая школа экономики (НИУ ВШЭ)}, Москва \\
Бакалавриат, Факультет компьютерных наук, КНАД \\
Сентябрь 2022 --- настоящее время

\section*{Курсы и учебные проекты}
\begin{itemize}[leftmargin=*]
  \item Летняя практика “Основы Deep Learning” — работа с PyTorch, построение CNN и RNN моделей.
  \item Алгоритмы и структуры данных (C++, Python)
  \item Курс по машинному обучению (базовый и продвинутый) — регрессия, классификация, ансамбли, детекция аномалий.
  \item \textbf{AI-проект:} разработка и внедрение модели для выявления диабетической ретинопатии по снимкам сетчатки глаза с использованием сверточных нейросетей.
  \item \textbf{ML-проект:} автогенерация обложек для видео на основе видеоряда и текстового описания — использование моделей Vision+Text (CLIP/BLIP) для автоматизации дизайна.
\end{itemize}

\section*{Навыки}
\begin{itemize}[leftmargin=*]
  \item \textbf{Языки программирования:} Python, C++
  \item \textbf{ML/AI:} sklearn, numpy, pandas, PyTorch
  \item \textbf{Инструменты:} Git, Docker, Linux, VSCode
\end{itemize}

\section*{Языки}
\begin{itemize}[leftmargin=*]
  \item Русский — свободно на уровне носителя
  \item Английский — B2 (читаю техническую документацию)
  \item Армянский — родной
\end{itemize}

\section*{О себе}
Ответственная, быстро учусь, люблю разбираться в новых технических темах. Сейчас активно изучаю области ML и системного программирования.

\end{document}

